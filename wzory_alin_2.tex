\documentclass{article}
\usepackage[landscape]{geometry}
\usepackage{url}
\usepackage{multicol}
\usepackage{amsmath}
\usepackage{esint}
\usepackage{amsfonts}
\usepackage{tikz}
\usetikzlibrary{decorations.pathmorphing}
\usepackage{amsmath,amssymb}

\usepackage{colortbl}
\usepackage{xcolor}
\usepackage{mathtools}
\usepackage{amsmath,amssymb}
\usepackage{enumitem}
\makeatletter
\usepackage[T1]{fontenc}
\usepackage[polish]{babel}
\usepackage[utf8]{inputenc}
\newcommand*\bigcdot{\mathpalette\bigcdot@{.5}}
\newcommand*\bigcdot@[2]{\mathbin{\vcenter{\hbox{\scalebox{#2}{$\m@th#1\bullet$}}}}}
\makeatother

\title{130 Cheat Sheet}

\advance\topmargin-.8in
\advance\textheight3in
\advance\textwidth3in
\advance\oddsidemargin-1.5in
\advance\evensidemargin-1.5in
\parindent0pt
\parskip2pt
\newcommand{\hr}{\centerline{\rule{3.5in}{1pt}}}
%\colorbox[HTML]{e4e4e4}{\makebox[\textwidth-2\fboxsep][l]{texto}
\begin{document}

\begin{center}{\huge{\textbf{Wzory Algebra Liniowa 2}}}\\
\end{center}
\begin{multicols*}{3}

\tikzstyle{mybox} = [draw=black, fill=white, very thick,
    rectangle, rounded corners, inner sep=10pt, inner ysep=10pt]
\tikzstyle{fancytitle} =[fill=black, text=white, font=\bfseries]

%------------ Heating ---------------
\begin{tikzpicture}
\node [mybox] (box){%
    \begin{minipage}{0.3\textwidth}
    	$\phi:V \rightarrow W$
    	\begin{align*}
    		\phi(v_1) &=
    		a_{11}w_1+a_{21}w_2+...+a_{m1}w_m\\
    		\phi(v_2) &=
    		a_{12}w_1+a_{22}w_2+...+a_{m2}w_m\\
    		&\vdots\\
    		\phi(v_n) &=
    		a_{1n}w_1+a_{2n}w_2+...+a_{mn}w_m
		\end{align*}

    	$M^V_W(\phi) =
    	\begin{bmatrix}
		a_{11} & a_{21} & ... & a_{m1}\\
		a_{12} & a_{22} & ... & a_{m2}\\
		\vdots\\
		a_{1n} & a_{2n} & ... & a_{mn}\\
		\end{bmatrix}$
    \end{minipage}
};
%------------ Heating Header ---------------------
\node[fancytitle, right=10pt] at (box.north west) {Macierz przekształcenia liniowego};
\end{tikzpicture}

%------------ Mixing ---------------
\begin{tikzpicture}
\node [mybox] (box){%
	\begin{minipage}{0.3\textwidth}
		\begin{align*}
			&\phi:V \rightarrow W, \quad \omega:W \rightarrow U \\
			&M^V_U(\phi \circ \omega)=M^W_U(\omega)\times M^V_U(\phi)\\
			&B=[b_{ij}]_{p\times n} i A=[a_{ij}]_{n\times m} \\
			&C=B*A=[c_{ij}]_{n \times m}, \quad c_{ij}=\sum^m_{k=1}b_{ik}a_{kj} \\
			&\textrm{$E$ - baza $V$, $F$ - baza $W$. Niech $v \in V$, $w \in W$. Wtedy} \\
			&\phi(v)=w \iff M^E_F(\phi)\times v_E=w_F \\
			&\textrm{gdzie $v_E$ - wektor $v$ z $V$ zapisany w bazie $E$}
		\end{align*}
	\end{minipage}
};
%------------ Mixing Header ---------------------
\node[fancytitle, right=10pt] at (box.north west) {Superpozycja przekształceń liniowych};
\end{tikzpicture}

%------------ Inner Product Spaces ---------------
\begin{tikzpicture}
\node [mybox] (box){%
	\begin{minipage}{0.3\textwidth}
		\begin{align*}
			&M^B_A(id) \textrm{ - macierz przejścia od bazy $A$ do bazy $B$} \\
			&M^B_A(id) \times v_B = v_A
		\end{align*}
		$$M^{A_2}_{B_2}(\phi) = 
		M^{B_2}_{B_1}(id)\times M^{A_1}_{B_1}(\phi)\times M^{A_2}_{A_1}(id)$$
	\end{minipage}
};
%------------ Inner Product Space Header ---------------------
\node[fancytitle, right=10pt] at (box.north west) {Macierz zmiany bazy};
\end{tikzpicture}

%------------ Gram-Schmidt Content ---------------
\begin{tikzpicture}
\node [mybox] (box){%
    \begin{minipage}{0.3\textwidth}
    fdsfmsą
		\begin{align*}
			&M^B_A(id) \textrm{ - macierz przejścia od bazy $A$ do bazy $B$} \\
			&M^B_A(id) \times v_B = v_A
		\end{align*}
		$$M^{A_2}_{B_2}(\phi) = 
		M^{B_2}_{B_1}(id)\times M^{A_1}_{B_1}(\phi)\times M^{A_2}_{A_1}(id)$$
    \end{minipage}
};
%------------ Gram-Schmidt Header ---------------------
\node[fancytitle, right=10pt] at (box.north west) {Dopełnienie algebraiczne i wyznacznik macierzy};
\end{tikzpicture}
%------------ Variation of Parameters Content ---------------------
\begin{tikzpicture}
\node [mybox] (box){%
    \begin{minipage}{0.3\textwidth}
    	\begin{align*}
        	F(x) &= y'' + y' \\
            y_h &= b_1y_1(x) + b_2y_2(x), y_1 y_2 \text{ are L.I.} \\
            y_p &= u_1(x)y_1(x) + u_2(x)y_2(x) \\
            u_1 &= \int^t -\frac{y_2F(t)dt}{w[y_1,y_2](t)} \\
            u_2 &= \int^t \frac{y_1F(t)dt}{w[y_1,y_2](t)} \\    		
            y &= y_h + y_p
    	\end{align*}
    \end{minipage}
};
%------------ Variation of Parameters Header ---------------------
\node[fancytitle, right=10pt] at (box.north west) {Variation of Parameters};
\end{tikzpicture}

%------------ ODE Content ---------------
\begin{tikzpicture}
\node [mybox] (box){%
    \begin{minipage}{0.3\textwidth}
    \small{
    	\begin{tabular}{lp{4cm} l}
		\textit{1st Order Linear} & Use integrating factor,
        \\ & $I = e^{\int P(x) dx}$ \\ \hline
		\textit{Separable:} & $ \int P(y) dy/dx = \int Q(x) $ \\ \hline
		\textit{HomogEnEous:} & $ dy/dx = f(x,y) = f(xt,yt) $ \\ &
        sub $ y = xV $ solve, then sub $ V = y/x $ \\ \hline
        \textit{Exact:} & If $ M(x,y) + N(x,y)dy/dx = 0 $ and $ M_y = N_x $ i.e. $ \langle M,N \rangle = \nabla F $ then $ \int_x M + \int_y N = F $ \\ \hline
        \textit{Order Reduction} & Let $ v = dy/dx $ then check other types \\ 
        &\textit{If purely a function of y, }$\frac{dv}{dx} = v\frac{dv}{dy}$\\
        \hline
        \textit{Variation of Parameters:} & When $y''+a_1y'+a_2y = F(x)$ \\
        & $F$ contains $\ln x$, $\sec x$, $\tan x$, $\div$ \\ \hline
        \textit{Bernoulli} & $y' + P(x)y = Q(x)y^n$ \\
        & $\div y^n$ \\
        &$y^{-n}y'+P(x)y^{1-n}=Q(x)$ \textit{Let }$U(x) = y^{1-n}(x)$ \\
        &$\frac{dU}{dx}=(1-n)y^{-n}\frac{dy}{dx}$ \\
        &$\frac{1}{1-n}\frac{du}{dx} + P(x)U(x) = Q(x)$ \textit{solve as a 1st order} \\ \hline
        \textit{Cauchy-Euler} &$x^ny^n + a_1x^{n-1}y^{n-1} + \cdots + a_{n-1}y^{n-2}+a_ny = 0$ \\
        &guess $y = x^r$ \\
        \textit{3 Cases:} \\
        \textit{1) Distinct real roots} &$y = ax^{r_1}+bx^{r_2}$ \\
        \textit{2) Repeated real roots} &$y = Ax^r + y_2$ \\
        &\textit{Guess} $y_2 = x^ru(x)$ \\
        &\textit{Solve for $u(x)$ and choose one ($A=1, C=0$)} \\
		\textit{3) Distinct complex roots} &$y=B_1x^a \cos (b \ln x) + B_2x^a\sin (b \ln x)$
	\end{tabular}}
    \end{minipage}
};
%------------ ODE Header ---------------------
\node[fancytitle, right=10pt] at (box.north west) {ODEs};
\end{tikzpicture}

%------------ Series Solution Content ---------------
\begin{tikzpicture}
\node [mybox] (box){%
    \begin{minipage}{0.3\textwidth}
    $y'' + p(x)y' + q(x)y = 0$ \\
    Useful when $p(x), q(x)$ not constant \\
    Guess $y = \sum_{n=0}^{\infty}a_n(x-x_0)^n$
    \small{
    	\begin{tabular}{lp{4cm} l}
        \hline
        $e^x$ & $\sum_{n=0}^{\infty} x^n/{n!}$ \\ \hline
        $\sin x$ & $\sum_{n=0}^{\infty} \frac{(-1)^n}{(2n+1)!}x^{2n+1}$ \\ \hline
        $\cos x$ & $\sum_{n=0}^{\infty} \frac{(-1)^n}{(2n)!}x^{2n}$ \\
	\end{tabular}}
    \end{minipage}
};
%------------ Series Solution Header ---------------------
\node[fancytitle, right=10pt] at (box.north west) {Series Solution};
\end{tikzpicture}

%------------ Systems of ODE Content ---------------
\begin{tikzpicture}
\node [mybox] (box){%
    \begin{minipage}{0.3\textwidth}
    \small{
    	\begin{tabular}{lp{4cm} l}
        $\vec{x}' = A\vec{x}$ \\
		\textit{A is diagonalizable} & $\vec{x}(t)=a_{1}e^{\lambda_1 t}\vec{v_1}+\cdots+ a_{n}e^{\lambda_n t}\vec{v_n}$ \\ \hline
        \textit{A is not diagonalizable} & $\vec{x}(t)=a_1e^{\lambda_1 t}\vec{v_1} + a_2e^{\lambda t}(\vec{w} + t\vec{v} )$ \\
        & where $(A - \lambda I)\vec{w} = \vec{v} $\\
        & $\vec{v}$ is an Eigenvector w/ value $\lambda$ \\
        & i.e. $\vec{w}$ is a generalized Eigenvector \\ \hline
        $\vec{x}' = A\vec{x} + \vec{B}$ &Solve $y_h$ \\
        & $\vec{x_1} = e^{\lambda_1t}\vec{v_1}, \vec{x_2} = e^{\lambda_2t}\vec{v_2}$ \\ 			& $\vec{X} = [\vec{x_1},\vec{x_2}]$ \\
        & $\vec{X}\vec{u}'=\vec{B}$ \\
        & $y_p = \vec{X}\vec{u}$ \\
        & $y = y_h + y_p$
	\end{tabular}}
    \end{minipage}
};
%------------ Systems of ODE Header ---------------------
\node[fancytitle, right=10pt] at (box.north west) {Systems};
\end{tikzpicture}

%------------ Exponentiation Content ---------------
\begin{tikzpicture}
\node [mybox] (box){%
    \begin{minipage}{0.3\textwidth}
    \small{
    	\begin{tabular}{lp{4cm} l}
        $A^n = SD^nS^{-1}$ \\
        \textit{D is the diagonalization of A}
	\end{tabular}}
    \end{minipage}
};
%------------ Spring-Mass Header ---------------------
\node[fancytitle, right=10pt] at (box.north west) {Matrix Exponentiation};
\end{tikzpicture}
\
%------------ Laplace Transforms Content ---------------
\begin{tikzpicture}
\node [mybox] (box){%
    \begin{minipage}{0.3\textwidth}
    $L[f](s) = \int_0^{\infty} e^{-sx}f(x)dx $\\
    ą
    \small{
    	\begin{tabular}{lp{4cm} l}
        $f(t) = t^n, n \geq 0 $ &$F(s) = \frac{n!}{s^{n+1}}, s > 0 $ \\
        $f(t) = e^{at}, a \textit{ constant}$ & $ F(s) = \frac{1}{s-a}, s > a$ \\
        $f(t) = \sin{bt}, b \textit{ constant}$ & $ F(s) = \frac{b}{s^2 + b^2}, s > 0$ \\
        $f(t) = \cos{bt}, b \textit{ constant}$ & $ F(s) = \frac{s}{s^2 + b^2}, s > 0$ \\
        $f(t) = t^{-1/2}$ & $F(s) = \frac{\pi}{s^{1/2}}, s > 0$ \\
        $f(t) = \delta(t-a)$ & $F(s) = e^{-as}$ \\
        $f'$ & $L[f'] = sL[f] - f(0)$ \\
        $f''$ & $L[f''] = s^2 L[f] - sf(0) - f'(0)$ \\
        $L[e^{at}f(t)]$ & $L[f](s-a)$ \\
        $L[u_a(t)f(t-a)]$ & $L[f]e^{-as}$ 
        \end{tabular}}
    \end{minipage}
};
%------------ Laplace Transforms Header ---------------------
\node[fancytitle, right=10pt] at (box.north west) {Laplace Transforms};
\end{tikzpicture}
%------------ Gaussian Integral Content ---------------------
\begin{tikzpicture}
\node [mybox] (box){%
    \begin{minipage}{0.3\textwidth}
	$\int_{-\infty}^{+\infty} e^{-1/2(\vec{x}^TA\vec{x})} = \frac{\sqrt{2\pi}^n}{\sqrt{\det A}}$
	\end{minipage}
};
%------------ Gaussian Integral Header ---------------------
\node[fancytitle, right=10pt] at (box.north west) {Gaussian Integral};
\end{tikzpicture}
\\
\\
\\
\\

%------------ Complex Numbers Content ---------------------
\begin{tikzpicture}
\node [mybox] (box){%
    \begin{minipage}{0.3\textwidth}
    \small{
        	\begin{tabular}{lp{4cm} l}
            \textit{Systems of equations} & If $\vec{w_1} = \vec{u(t)} + i\vec{v(t)}$ is a solution, $\vec{x_1} = \vec{u(t)}, \vec{x_2} = \vec{v(t)}$ are solutions \\ 
            & i.e. $\vec{x_h} = c_1 \vec{x_1} + c_2 \vec{x_2}$ \\
            \hline
            \textit{Euler's Identity} &$e^{ix} = \cos x + i \sin x$
			\end{tabular}
    }
	\end{minipage}
};
%------------ Gaussian Integral Header ---------------------
\node[fancytitle, right=10pt] at (box.north west) {Complex Numbers};
\end{tikzpicture}

%------------ Vector Spaces ---------------
\begin{tikzpicture}
\node [mybox] (box){%
    \begin{minipage}{0.3\textwidth}
    $v_1, v_2 \in V$\\
    1. $v_1 + v_2 \in V$ \\
	2. $k \in \mathbb{F}, kv_1 \in V $ \\
	3. $ v_1 + v_2 = v_2 + v_1 $ \\
	4. $(v_1 + v_2) + v_3 = v_1 + (v_2 + v_3) $ \\
	5. $\forall v \in V, 0 \in V \mid 0 + v_1 = v_1 + 0 = v_1$ \\
    6. $\forall v \in V, \exists -v \in V \mid v + (-v) = (-v) + v = 0 $ \\
    7. $\forall v \in V, 1 \in \mathbb{F} \mid 1*v = v$ \\
    8. $\forall v \in V, k,l \in \mathbb{F}, (kl)v = k (lv)$ \\
    9. $\forall k \in \mathbb{F}, k(v_1 + v_2) = kv_1 + kv_2$ \\
    10. $\forall v \in V, k,l \in \mathbb{F}, (k+l)v = kv + lv$
    \end{minipage}
};
%------------ Vector Space Header ---------------------
\node[fancytitle, right=10pt] at (box.north west) {Vector Spaces};
\end{tikzpicture}
\end{multicols*}
\end{document}


Contact GitHub API Training Shop Blog About
© 2016 GitHub, Inc. Terms Privacy Security Status Help